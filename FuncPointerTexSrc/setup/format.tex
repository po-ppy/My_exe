% 重定义强调字体的代码,解决默认强调字体是italic,此时中文会用楷体代替,
% 在此设置为加粗,注意需要使用etoolbox宏包
\makeatletter
\let\origemph\emph
\newcommand*\emphfont{\normalfont\bfseries}
\DeclareTextFontCommand\@textemph{\emphfont}
\newcommand\textem[1]{%
  \ifdefstrequal{\f@series}{\bfdefault}
    {\@textemph{\CTEXunderline{#1}}}
    {\@textemph{#1}}%
}
\RenewDocumentCommand\emph{s o m}{%
  \IfBooleanTF{#1}
    {\textem{#3}}
    {\IfNoValueTF{#2}
      {\textem{#3}\index{#3}}
      {\textem{#3}\index{#2}}%
     }%
}
\makeatother   

% ========设置标题的格式========
\ctexset{
  section = {
    format+ = \zihao{-4} \heiti \raggedright,
    name = {,、},
    number = \chinese{section},
    beforeskip = 1.0ex plus 0.2ex minus .2ex,
    afterskip = 1.0ex plus 0.2ex minus .2ex,
    aftername = \hspace{0pt}
  },
  subsection = {
    format+ = \zihao{5} \heiti \raggedright,
    % name={\thesubsection、},
    name = {,、},
    number = \arabic{subsection},
    beforeskip = 1.0ex plus 0.2ex minus .2ex,
    afterskip = 1.0ex plus 0.2ex minus .2ex,
    aftername = \hspace{0pt}
  }
}

% % ========设置标题的格式========
% \ctexset{
%   section = {
%     format+ = \zihao{-4}  \raggedright,
%     name = {,、},
%     number = \chinese{section},
%     beforeskip = 1.0ex plus 0.2ex minus .2ex,
%     afterskip = 1.0ex plus 0.2ex minus .2ex,
%     aftername = \hspace{0pt}
%   }
% }

% ========不需要页眉=======
\pagestyle{plain}

\labelformat{figure}{\figurename#1}
\labelformat{table}{\tablename#1}

% 设置插图路径
\graphicspath{{figures/}}

% 定义提醒字体
\newcommand{\alert}[1]{\textcolor{red}{\textbf{#1}}}

% 定义引号命令
\newcommand{\qtmark}[1]{``#1''}

% 定义专有名词
\newcommand{\cb}{\texttt{Code::Blocks}}
\newcommand{\mww}{\texttt{MinGW-w64}}
\newcommand{\mfile}{\qtmark{\texttt{Makefile}}}
%%%%%%%%%%% 定义颜色%%%%%%%%%%%%%%%%%%
\definecolor{listinggray}{gray}{0.92}

% 定义C语言代码参数
\setminted{fontsize=\small, mathescape, breaklines=true,
  breakautoindent=false, autogobble}
\newmintinline{cpp}{fontsize=\normalsize}
\newmintinline{shell}{fontsize=\normalsize}
\newminted{cpp}{bgcolor=yellow!20!white, mathescape, autogobble,fontsize=\small, frame=lines}
\newminted[cpptt]{cpp}{bgcolor=yellow!20!white, mathescape,
  autogobble,fontsize=\small,escapeinside=||, frame=lines}
\newminted[shell]{sh}{autogobble,fontsize=\small, frame=lines}
\newmintedfile{cpp}{linenos=true, fontsize=\small}

% 定义makefile代码参数
\newmintinline{basemake}{fontsize=\normalsize}
\newmintinline[makefileinline]{basemake}{fontsize=\normalsize,escapeinside=!!}
\newminted{basemake}{mathescape,
  autogobble,fontsize=\small,frame=lines}
\newminted[makefilett]{basemake}{mathescape,
  autogobble,fontsize=\small,frame=lines,escapeinside=!!}
\newmintedfile{basemake}{mathescape, autogobble,fontsize=\small,frame=lines}

%bgcolor=listinggray, 

% 参考文献
\bibliographystyle{plain}

%%% Local Variables:
%%% mode: latex
%%% TeX-master:"../main.tex"
%%% End:
