% =========数学符号宏包=========
\usepackage{amssymb}
\usepackage{amsmath}

% =========设置页边距宏包=========
\usepackage[left=2.8cm,right=2.5cm,top=2.54cm,bottom=2.54cm]{geometry}

% 添加附件的宏包
\usepackage[author=耿楠,
%scale=0.35,
color={0 0 1.0},%color=red,
%mimetype=text/plain,
subject=源代码,
description=打开或下载该源代码,
icon=Paperclip]{attachfile2}

% =========超链接宏包=========
\usepackage{hyperref}
\hypersetup{ % 取消超链接文字的包围框显示
    colorlinks=false,
    pdfborder={0 0 0},
}

% =========浮动体增强宏包=========
\usepackage{floatrow}
\floatsetup[figure]{objectset=centering, margins=centering}

% ========排版键盘组合和菜单的宏包=========
\usepackage{menukeys}

% ========排版代码的宏包=========
\usepackage{minted}

% ========处理标题的宏包=========
\usepackage[labelsep=quad]{caption}
\usepackage{varioref}

% =========颜色宏包=========
\usepackage{xcolor}

% =========提供了ltx2e里面命令、环境的一些补丁=========
\usepackage{etoolbox}

% =========解决minted包排版代码的跨页问题=========
\usepackage[linecolor=black, topline=true, bottomline=true,
leftline=false, rightline=false,
backgroundcolor=yellow!20!white]{mdframed}


%%% Local Variables:
%%% mode: latex
%%% TeX-master:"../main.tex"
%%% End:
